%
% moretype.sty -- Typography
%

% Avoid widows, orphans, and lines sticking into the margin
\clubpenalty=500
\widowpenalty=1000
\tolerance=300

% Let LaTeX hyphenate as much as it needs to (for better justification)
\hyphenpenalty=0

% Set a point size of 11.75 and leading of 15pt for the body text
\renewcommand\Large{\@setfontsize\Large{16pt}{20pt}}
\renewcommand\large{\@setfontsize\large{15pt}{18pt}}
\renewcommand\normalsize{\@setfontsize\normalsize{11.75pt}{15pt}}
\renewcommand\small{\@setfontsize\small{10pt}{13pt}}

% Create \newthought and \longnewthought commands for starting a new section
% without a section heading. It adds a little vertical space and sets three or
% four words in all small caps.
\def\newthought #1 #2 #3 {\vspace{1\baselineskip plus 5pt minus 2pt}\noindent {\lowercaps{#1 #2 #3}} }
\def\longnewthought #1 #2 #3 #4 {\vspace{1\baselineskip plus 5pt minus 2pt}\noindent {\lowercaps{#1 #2 #3 #4}} }

% Create a command for finishing off a long document
\newcommand{\finalfleuron}{
  \vspace{\baselineskip}
  \centering
  \addfontfeature{Ornament=4}
  \large\textbullet\normalsize\normalfont\
}

% Add an alternate one
\newcommand{\altfinalfleuron}{
  \vspace{\baselineskip}
  \centering
  \addfontfeature{Ornament=3}
  \large\textbullet\normalsize\normalfont\
}

% Microtypographic awesomeness!
\RequirePackage[final]{microtype}

% Commands for letterspaced capitals
\newcommand{\lowercaps}[1]{\textsc{\addfontfeature{Letters={UppercaseSmallCaps}}#1}}
\newcommand{\uppercaps}[1]{
  \addfontfeature{
    Letters={Uppercase},
    Kerning={Uppercase},
    LetterSpace=15
  }{\MakeTextUppercase{#1}}
}

% Command for abbreviations and acronyms
\newcommand{\abbr}[1]{\lowercaps{#1}}

% Command for fancier ampersand
\newcommand{\oldand}{\textit{\&}}

% Allow loading system fonts
\RequirePackage[no-math]{fontspec}

% Settings that apply to all fonts
\defaultfontfeatures{
  SmallCapsFeatures={               % Letterspace small caps
    LetterSpace=5.0
  },
  % Path to where installfonts.sh copies fonts from Adobe Fonts
  Path={/Users/aramis/.local/share/polytextum/fonts/},
  % All of Adobe Fonts is .otf, so let’s declare that
  Extension=.otf
}

% Scala Sans from Adobe Fonts
% We set scale to one here because Arno is pretty small in its font file
% Proportional text figures are default here
\setsansfont[
  ItalicFont=*-Ita,
  BoldFont=*-Bold,
  BoldItalicFont=*-BoldIta,
  Scale=1
]{ScalaSansPro}

% Arno Pro from Adobe Fonts
\setromanfont[
  % Regular settings
  Numbers={OldStyle, Proportional}, % Use proportional text figures
  Scale=MatchLowercase,             % Match x-height to Scala Sans
  % Optical sizes
  UprightFont=*-Regular,
  UprightFeatures={
    SizeFeatures={
      {Size={-8.4},     Font=*-Caption},
      {Size={8.5-10.9}, Font=*-SmText},
      {Size={11-13.9}},
      {Size={14-21.4},  Font=*-Subhead},
      {Size={21.5-},    Font=*-Display}
    },
  },
  ItalicFont=*-Italic,
  ItalicFeatures={
    SizeFeatures={
      {Size={-8.4},     Font=*-ItalicCaption},
      {Size={8.5-10.9}, Font=*-ItalicSmText},
      {Size={11-13.9}},
      {Size={14-21.4},  Font=*-ItalicSubhead},
      {Size={21.5-},    Font=*-ItalicDisplay}
    },
  },
  BoldFont=*-Smbd,
  BoldFeatures={
    SizeFeatures={
      {Size={-8.4},     Font=*-SmbdCaption},
      {Size={8.5-10.9}, Font=*-SmbdSmText},
      {Size={11-13.9}},
      {Size={14-21.4},  Font=*-SmbdSubhead},
      {Size={21.5-},    Font=*-SmbdDisplay}
    },
  },
  BoldItalicFont=*-SmbdItalic,
  BoldItalicFeatures={
    SizeFeatures={
      {Size={-8.4},     Font=*-SmbdItalicCaption},
      {Size={8.5-10.9}, Font=*-SmbdItalicSmText},
      {Size={11-13.9}},
      {Size={14-21.4},  Font=*-SmbdItalicSubhead},
      {Size={21.5-},    Font=*-SmbdItalicDisplay}
    },
  },
]{ArnoPro}

% Nexus Typewriter Pro, from (you guessed it!) Adobe Fonts
% Monospaced text figures are default here
\setmonofont[
  UprightFont=*,
  ItalicFont=*-Italic,
  BoldFont=*-Bold,
  BoldItalicFont=*-BoldItalic,
  Scale=MatchLowercase  % Match size to Scala Sans
]{NexusTypewriterPro}

% Set up swash italics
\DeclareTextFontCommand{\textsw}{
  \itshape\addfontfeature{
    Style=Swash,
    Contextuals=Swash
  }
}

% Generate math fonts from system fonts
\RequirePackage[italic]{mathastext}

% Use tabular figures in some places
\RequirePackage[ % Use them for ...
  eqno,          % equation numbers
  enum,          % enumerate item labels
  bib,           % bibliography item labels
  lineno         % line numbers
]{tabfigures}

% Same amount of space between sentences as between words
\frenchspacing

% Intelligently suppress ligatures
\RequirePackage{selnolig}

% Suppress purely ornamental ligaures
\nolig{Th}{T|h}
\nolig{TT}{T|T}
\nolig{RA}{R|A}
\nolig{KA}{K|A}

% Allow customsation of (the typography of (the content of floats))
\let\newfloat\undefined
\RequirePackage{floatrow}

% Use sanserif for figure and table content
\DeclareFloatFont{polytextum}{\sffamily}
\floatsetup[figure]{font=polytextum}
\floatsetup[table]{font=polytextum}
