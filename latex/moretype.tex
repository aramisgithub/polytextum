%
% moretype.sty -- Typography
%

% Avoid widows, orphans, and lines sticking into the margin
\clubpenalty=500
\widowpenalty=1000
\pretolerance=90
\tolerance=250

% Discourage hyphenating across pages
\brokenpenalty=500

% Let LaTeX hyphenate as much as it needs to (for better justification)
\hyphenpenalty=0
\doublehyphendemerits=0

% Set a base point size of 12 on a leading of 16pt
\renewcommand\Large{\@setfontsize\Large{16bp}{20bp}}
\renewcommand\large{\@setfontsize\large{15bp}{18bp}}
\newcommand\smidgelarger{\@setfontsize\normalsize{12.5bp}{16bp}}
\renewcommand\normalsize{\@setfontsize\normalsize{12bp}{16bp}}
\newcommand\smidgesmaller{\@setfontsize\normalsize{11.5bp}{16bp}}
\renewcommand\small{\@setfontsize\small{10bp}{14bp}}

% Remove all indications of a new paragraph
\newcommand{\stoppar}{\noindent\ignorespaces}

% A command for starting a new section without a section heading. This is done
% by adding an asterisk centered both vertically and horizontally over two
% lines.
\newcommand{\sbcorrection}{0.5em}
\newcommand{\sectionbreak}{
  \vspace{\baselineskip}
  \vspace{-\sbcorrection}
  {\centering *\\}
  \vspace{\sbcorrection}
  \par\stoppar
}

% Use \bigpar to create a new paragraph with an empty line in between instead of
% an indent. This is useful when you feel that one paragraph isn't enough
% separation, and \sectionbreak is too much.
\newcommand{\bigpar}{\vskipnextgrid\stoppar}

% Create \newthought and \longnewthought commands for a rather dramatic and
% unusual way to start a new section without a section heading. It is probably
% too regal for most uses, but I'll keep it because it may come in handy. It
% adds a little vertical space and sets three or four words in all small caps.
\def\newthought #1 #2 #3 {\vspace{1\baselineskip plus 5bp minus 2bp}\noindent {\lowercaps{#1 #2 #3}} }
\def\longnewthought #1 #2 #3 #4 {\vspace{1\baselineskip plus 5bp minus 2bp}\noindent {\lowercaps{#1 #2 #3 #4}} }

% Create a command for finishing off a long document
\newcommand{\finalfleuron}{
  \vspace{\baselineskip}
  \centering
  \addfontfeature{Ornament=6}
  \large\textbullet\normalsize\normalfont\
}

% Add an alternate one
\newcommand{\altfinalfleuron}{
  \vspace{\baselineskip}
  \centering
  \addfontfeature{Ornament=12}
  \large\textbullet\normalsize\normalfont\
}

% Microtypographic awesomeness!
\RequirePackage[
  final,      % Activate even with draft class option
  stretch=20, % Let font expansion stretch the font ±2%, a fifth of what the
  shrink=20   % word space can be stretched and shrunk (±10%). We use a fifth
              % because, in English, there are five times more letters
              % than spaces
]{microtype}

% Commands for letterspaced capitals
\newcommand{\lowercaps}[1]{\textsc{\MakeTextLowercase{#1}}}
\newcommand{\uppercaps}[1]{{\caps\MakeTextUppercase{#1}}}

% Command for abbreviations and acronyms
\newcommand{\abbr}[1]{\lowercaps{#1}}

% Command for fancier ampersand
\newcommand{\oldand}{\textit{\&}}

% Allow loading system fonts
\RequirePackage[no-math]{fontspec}

% Settings that apply to all fonts
\defaultfontfeatures{
  % Letterspace small caps
  SmallCapsFeatures={ %
    LetterSpace=3.0
  },
  % Path to where installfonts.sh copies fonts from Adobe Fonts
  Path={/Users/aramis/.local/share/polytextum/fonts/},
  % All of Adobe Fonts is .otf, so let’s declare that
  Extension=.otf,
  % We want, in general, for LaTeX to rely more on methods other than changing
  % the word space for justification, so we let it stretch and shrink ±10%,
  % which is less than is allowed by default. In additition, the normal size of
  % the word space is reduced by 5%.
  WordSpace={0.95,1.045,0.855}
}

% Minion 3 from Adobe Fonts
\setmainfont[
  % Regular settings
  Scale=1,                          % Base the size of other fonts on Minion
  Numbers={OldStyle, Proportional}, % Use proportional text figures
  % Optical sizes
  UprightFont=Minion3-Regular,
  UprightFeatures={
    SizeFeatures={
      {Size={-8.9},     Font=Minion3Caption-Regular},
      {Size={10-10.9},  Font=Minion3-Medium},
      {Size={11-13.9}},
      {Size={14-23.9},  Font=Minion3Subhead-Regular},
      {Size={24-},      Font=Minion3Display-Regular}
    },
  },
  ItalicFont=Minion3-Italic,
  ItalicFeatures={
    SizeFeatures={
      {Size={-8.9},     Font=Minion3Caption-Italic},
      {Size={10-10.9},  Font=Minion3-MediumItalic},
      {Size={11-13.9}},
      {Size={14-23.9},  Font=Minion3Subhead-Italic},
      {Size={24-},      Font=Minion3Display-Italic}
    },
  },
  BoldFont=Minion3-Semibold,
  BoldFeatures={
    SizeFeatures={
      {Size={-8.9},    Font=Minion3Caption-Semibold},
      {Size={9-13.9}},
      {Size={14-23.9}, Font=Minion3Subhead-Semibold},
      {Size={24-},     Font=Minion3Display-Semibold}
    },
  },
  BoldItalicFont=Minion3-SemiboldItalic,
  BoldItalicFeatures={
    SizeFeatures={
      {Size={-8.9},    Font=Minion3Caption-SemiboldItalic},
      {Size={9-13.9}},
      {Size={14-23.9}, Font=Minion3Subhead-SemiboldItalic},
      {Size={24-},     Font=Minion3Display-SemiboldItalic}
    },
  }
]{Minion3}

% Seravek, comes built in on macOS
\setsansfont[
  Scale=MatchLowercase,             % Match size to Minion
  Numbers={OldStyle, Proportional}, % Use proportional text figures
  Path={/Library/Fonts/},
  Extension=.ttc
]{Seravek}

% Fira Mono from Adobe Fonts
% Monospaced text figures are default here
\setmonofont[
  Scale=MatchLowercase,  % Match size to Minion
  Numbers={OldStyle},    % Use text figures
  UprightFont=*,
  BoldFont=*-Bold
]{FiraMonoOT}

% Special caps font family with extra word spacing and some other adjustments
\newfontfamily\caps[
  % Regular settings
  Scale=MatchLowercase,           % Match to standard Minion
  Numbers={Lining, Proportional}, % Use proportional lining figures
  Letters={Uppercase},            % A few small adjustments for all-caps text
  LetterSpace=15,                 % Lots of letterspacing
  WordSpace={2,2.1,1.9},          % A far wider word space
  % Optical sizes
  UprightFont=Minion3-Regular,
  UprightFeatures={
    SizeFeatures={
      {Size={-8.9},     Font=Minion3Caption-Regular},
      {Size={10-10.9},  Font=Minion3-Medium},
      {Size={11-13.9}},
      {Size={14-23.9},  Font=Minion3Subhead-Regular},
      {Size={24-},      Font=Minion3Display-Regular}
    },
  },
  ItalicFont=Minion3-Italic,
  ItalicFeatures={
    SizeFeatures={
      {Size={-8.9},     Font=Minion3Caption-Italic},
      {Size={10-10.9},  Font=Minion3-MediumItalic},
      {Size={11-13.9}},
      {Size={14-23.9},  Font=Minion3Subhead-Italic},
      {Size={24-},      Font=Minion3Display-Italic}
    },
  },
  BoldFont=Minion3-Semibold,
  BoldFeatures={
    SizeFeatures={
      {Size={-8.9},    Font=Minion3Caption-Semibold},
      {Size={9-13.9}},
      {Size={14-23.9}, Font=Minion3Subhead-Semibold},
      {Size={24-},     Font=Minion3Display-Semibold}
    },
  },
  BoldItalicFont=Minion3-SemiboldItalic,
  BoldItalicFeatures={
    SizeFeatures={
      {Size={-8.9},    Font=Minion3Caption-SemiboldItalic},
      {Size={9-13.9}},
      {Size={14-23.9}, Font=Minion3Subhead-SemiboldItalic},
      {Size={24-},     Font=Minion3Display-SemiboldItalic}
    },
  }
]{Minion3}

% Set up swash italics
\DeclareTextFontCommand{\textsw}{
  \itshape\addfontfeature{
    Style=Swash,
    Contextuals=Swash
  }
}

% Generate math fonts from system fonts
\RequirePackage[italic]{mathastext}

% Use tabular figures in some places
\RequirePackage[ % Use them for ...
  eqno,          % equation numbers
  enum,          % enumerate item labels
  bib,           % bibliography item labels
  lineno         % line numbers
]{tabfigures}

% Same amount of space between sentences as between words
\frenchspacing

% Allow customsation of (the typography of (the content of floats))
\let\newfloat\undefined
\RequirePackage{floatrow}

% Use sanserif for figure and table content
\DeclareFloatFont{polytextum}{\sffamily}
\floatsetup[figure]{font=polytextum}
\floatsetup[table]{font=polytextum}
